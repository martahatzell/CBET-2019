
\renewcommand{\LeftFooter}{Biosketch}
\renewcommand{\LeftHeader}{Marta C. Hatzell}
\renewcommand{\PageLimit}{2}
\begin{center}
{\bf Marta C. Hatzell Biosketch} \\*[3mm]
\end{center}

\setcounter{section}{0}
\section{Professional Preparation}\label{MCH-professional-preparation}

\begin{itemize}
\itemsep1pt\parskip0pt\parsep0pt
\item
  Pennsylvania State University, University Park, Pennsylvania

  \begin{itemize}
  \itemsep1pt\parskip0pt\parsep0pt
  \item
    B.S., Mechanical Engineering, 2009
  \end{itemize}
\item
  Pennsylvania State University, University Park, Pennsylvania

  \begin{itemize}
  \itemsep1pt\parskip0pt\parsep0pt
  \item
    MS, Mechanical Engineering, 2010
  \end{itemize}
  \item
  Pennsylvania State University, University Park, Pennsylvania

  \begin{itemize}
  \itemsep1pt\parskip0pt\parsep0pt
  \item
    MS, Environmental Engineering, 2014
  \end{itemize}
\item
  Pennsylvania State University, University Park, Pennsylvania

  \begin{itemize}
  \itemsep1pt\parskip0pt\parsep0pt
  \item
    Ph.D, Mechanical Engineering, 2014
  \end{itemize}
\item
  University of Illinois, Urbana Champaign, Illinois

  \begin{itemize}
  \itemsep1pt\parskip0pt\parsep0pt
  \item
    Postdoc, Material Science and Engineering, 2014-2015
  \end{itemize}
\end{itemize}

\section{Appointments}\label{MCH-appointments}

\begin{itemize}
\itemsep1pt\parskip0pt\parsep0pt
\item
  \emph{Assistant Professor, 08/15/2015- Present}

  \begin{itemize}
  \itemsep1pt\parskip0pt\parsep0pt
  \item
    Georgia Institute of Technology, George W. Woodruff School of Mechanical Engineering
  \end{itemize}
\end{itemize}

\section{Products most closely related to the proposed
project:}\label{MCH-products-most-closely-related-to-the-proposed-project}

\begin{enumerate}
\def\labelenumi{\arabic{enumi}.}
\item
Medford, Andrew J., and Marta C. Hatzell. "Photon-Driven Nitrogen Fixation: Current Progress, Thermodynamic Considerations, and Future Outlook." ACS Catalysis 7, no. 4 (2017): 2624-2643.
\item
Zhu, X., Hatzell, M. C., and Logan, B. E. (2014). Microbial reverse-electrodialysis electrolysis and chemical-production cell for H2 production and CO2 sequestration. Environmental science and technology letters, 1(4), pp.231-235.
\item
Hatzell, M. C., Turhan, A., Kim, S., Hussey, D. S., Jacobson, D. L., and Mench, M. M. (2011). Quantification of temperature driven flow in a polymer electrolyte fuel cell using high-resolution neutron radiography. Journal of The Electrochemical Society, 158(6), pp.B717-B726.
\item
Hatzell, M. C., Raju, M., Watson, V. J., Stack, A. G., van Duin, A. C., and Logan, B. E. (2014). Effect of strong acid functional groups on electrode rise potential in capacitive mixing by double layer expansion. Environmental science and technology, 48(23), pp.14041-14048.
\item
Zhu, Xiuping, Matthew D. Yates, Marta C. Hatzell, Hari Ananda Rao, Pascal E. Saikaly, and Bruce E. Logan. "Microbial community composition is unaffected by anode potential." Environmental science \& technology 48, no. 2 (2014): 1352-1358.
\end{enumerate}

\section{Other significant products:}\label{MCH-other-significant-products}

\begin{enumerate}
\def\labelenumi{\arabic{enumi}.}
\item
Hatzell, M. C., Cusick, R. D., and Logan, B. E. (2014). Capacitive mixing power production from salinity gradient energy enhanced through exoelectrogen-generated ionic currents. Energy and Environmental Science, 7(3), pp.1159-1165.
\item
Geise, G. M., Curtis, A. J., Hatzell, M. C., Hickner, M. A., and Logan, B. E. (2013). Salt concentration differences alter membrane resistance in reverse electrodialysis stacks. Environmental Science and Technology Letters, 1(1), pp 36-39.
\item
Kim, Y., Hatzell, M. C., Hutchinson, A. J., and Logan, B. E. (2011). Capturing power at higher voltages from arrays of microbial fuel cells without voltage reversal. Energy and Environmental Science, 4(11), 4662-4667.
\item
Hatzell, K. B., Hatzell, M.C., Cook, K.M., Boota, M, Housel, G.M., McBride, A., Kumbur, E.C. and Gogotsi, Y. (2015) Effect of oxidation of carbon material on suspension electrodes for flow electrode capacitive deionization. Environmental science and technology 49,(5), pp 3040-3047.
\item
Hatzell, M. C., Hatzell, K. B., and Logan, B. E. (2014). Using flow electrodes in multiple reactors in series for continuous energy generation from capacitive mixing. Environmental Science and Technology Letters, 1(12), pp 474-478.
\end{enumerate}

\section{Synergistic Activities}\label{MCH-synergistic-activities}

\begin{enumerate}
\def\labelenumi{\arabic{enumi}.}
\item
A Food-Energy-Water (FEW) fellow within Georgia Institute of Technologies - Serve-Learn-Sustain program. Helping to develop relationships between GT and our community through FEW topics. 
\item
Chaired section at the Electrochemical Society conference on the Water-Energy Nexus (2013, 2016, 2017)
  \item
 On the Organizing committee for symposium on Water related technologies at ACS Fall 2018. 
  \item
Class of 1969 Teaching fellow at Georgia Tech- exploring teaching best practices. 
  \item
Host atlanta area teachers through the GIFT program in my lab to work on FEW related projects during the summer (2016, 2017) 
\end{enumerate}