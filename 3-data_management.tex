\renewcommand{\LeftFooter}{Data Management Plan}
\renewcommand{\PageLimit}{2}

\begin{center}
{\bf Data Management Plan} \\*[3mm]
\end{center}

\setcounter{section}{0}

\section{Overview}
The data management plan (DMP) is for the project ``\@title''. The plan requires no additional cost, as data are stored on the computer where the data originated, backed up to a central server specifically meant for data storage and sharing amongst PI’s, and ultimately disseminated through publications and open databases that are free to the public.

\section{Capture}
\subsection{Experimental data}
The characterization and testing of catalytic samples will lead to data of the following types: gas chromatography traces, infrared and ultraviolet-visible spectra, X-ray photolectron spectra, X-ray absorption spectra, X-ray diffraction patterns, nitrogen physisorption isotherms, elemental analysis reports, transmission and scanning electron micrographs, EDS and EELS maps, mass spectrometry data, temperature programmed desorption data. The data is typically obtained as an output file from the software that operates an instrument. In some cases, files will be received from collaborators or commercial service laboratories. In order to facilitate collaboration amongst the PI’s these files will be periodically transferred to a central server in the PACE supercomputing cluster.

\subsection{Computational data}
Data generated by computational simulations will automatically be captured on the PACE cluster.

\subsection{Metadata}
Capture and storage of metadata is critical to ensure the reproducibility and long-term value of the data. Both experimental and computational data will have associated meta-data. The experimental metadata will mainly consist of experimental details of catalyst preparation, reaction procedures, and characterization procedures, while computational meta-data will consist of the necessary inputs and scripts to reproduce the data. The metadata will be initially captured in standard or electronic laboratory notebooks, and transferred to a standard digital format (json) when the data is transferred to the central server. The json files containing pertinent metadata will be permanently stored with the raw data.

\subsection{Physical samples}
Samples will be stored and catalogued according to established procedures in the Hatzell group. The standardized format will enable data retrieval by project, date, or conditions. Samples typically do not have an indefinite shelf life, and will be disposed 5 years after the project ends.

\section{Analysis}
Data analysis is critical to extracting useful knowledge from raw data. For this project, the raw experimental data will be processed to determine various characteristics of catalysts, such as surface areas, dispersions of supported species, and the nature and concentration of surface species, and raw computational data will be analyzed to determine key quantities such as surface, adsorption, and activation energies, charge distributions, and band gaps. Some analysis workflows are standardized and included in commercially available software packages, while others are customized and evolving. When standardized analysis techniques are used this will be included as metadata and stored with the results of the analysis. Customized workflows will also be captured by storing the necessary scripts along with the analysis results.

\section{Sharing and Dissemination}

\subsection{Data Sharing}
Sharing data, metadata, and analysis workflows amongst the Hatzell, Tang, and Medford groups will enable efficient collaboration. A central server within the PACE cluster will be used to facilitate easy access to shared data, and metadata capture will ensure that shared data has the necessary context to be useful. Furthermore, the MATIN e-Collaboration platform currently under development at Georgia Tech will be explored as a route to share intermediate results, workflows, and discussions via “research blogs”. These blogs will initially be private to ensure data security, but can easily be opened to the public once results have been published.

\subsection{Data Sharing and Dissemination}
The broad dissemination of research data as fundamental to progress in science. As such, published data generated within this proposal will be made freely and publicly available in compliance with the data sharing directive issued by the Office of Science and Technology Policy and relevant funding agencies. Specifically, publishable data and associated metadata will be uploaded to Citrination, the web-based data platform for materials, chemical, and device information. Citrine Informatics will host these data at no cost to this contract, or the government, for public access. In addition, computational reaction energies/barriers will be uploaded to the CatApp platform for broad dissemination within the computational catalysis community. Data shared will include: experimental procedures, material characterization, analysis of surface reactions, performance of catalysts, and adsorption/activation energies.  Data will also be available through journal publications – either the article or the supplemental information - and completed theses.


\section{Preservation}
\subsection{Storage and Backup during the Project}
Data will be stored locally on the computer of the data origination and will be periodically transferred securely over the campus network to a central server for longer-term data retention. Similarly, data from experiments at national laboratories will be copied onto a USB drive at the end of a trip and transferred to the server at GT. The fact that the central server will be used to share data amongst the PI’s will ensure that relevant data is transferred to the server at an appropriate frequency. The network server is a dedicated server that is backed up automatically. The data on the server is preserved for at least a period of seven (7) years where it is locally accessible, and then the data is preserved offline for three (3) more years, accessible with assistance from local network specialists. 

\subsection{Data Storage Duration} Digital data and metadata will be retained for ten (10) years on the centralized server. Physical laboratory notebooks will be stored by each faculty advisor for at least five (5) years. The most critical data will be widely disseminated, ensuring that it is available through publication outlets and open databases on a permanent basis.

\section{Protection}
All unpublished data will be stored on computers within secure campus networks. At this time, no additional security elements should be required.

\section{Rationale}
Georgia Tech commits to provide server space for the results from the proposed work for at least 10 years. The costs are covered as part of the requested indirect costs. The use of publication outlets and open databases provides a cost-effective approach to ensure that the most critical data are available on a permanent basis.
