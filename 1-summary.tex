\renewcommand{\LeftFooter}{Project Summary}
\renewcommand{\PageLimit}{1} %0 page limit puts no "of X" in page numbers
\vspace{-15pt}

\textbf{Summary:} In natural environments, a range of nanominerals, dissolved metals and organic compounds contribute to the natural abiotic chemical nitrogen cycle. Nanomaterial based semiconductors are widely abundant in global sands and soils. The role these natural nanominerals play in the nitrogen cycle is unknown. Throughout the last century significant evidence has been presented suggesting that  abiotic nitrogen-based transformations are significant. Specifically, abiotic solar driven processes such as nitrogen reduction and oxidation, ammonia oxidation and nitrate reduction have all been demonstrated on environmentally abundant minerals and metals. These processes are central to the nitrogen cycle, and are analogous to the well-studied biological processes: nitrogen fixation, nitrification and denitrification. Furthermore, these abiotic transformations have all been demonstrated  under environmentally relevant conditions (ambient pressure and temperature).  Here, through the use of first-principles calculations, photocatalytic testing, and in situ spectroscopic tools, the chemical mechanisms and physical phenomena that enable gas-phase photocatalytic nitrogen fixation and ammonia oxidation environmental settings. This fundamental understanding will provide insight into the role of terrestrial sands and minerals on the natural nitrogen cycle, providing a foundation for more accurate models of nutrient fluxes in terrestrial settings. Furthermore, the findings will elucidate the interactions between oxide surfaces, photons, and nitrogen compounds that enable these transformations. This knowledge will lay the groundwork for future technologies to manage and control the nitrogen cycle in a more sustainable way.

Intensification of agriculture and rising emissions from the burning of fossil fuels has led to a 3-5x increase in reactive nitrogen emissions (N$_2$O,NH$_3$,NO$_x$) over the past century. 

\textbf{Intellectual Merit:} The work will seek to unambiguously determine the role of natural nano-minerals play in enabling key photocatalytic nitrogen-based transformation (nitrogen fixation, nitrification, and ammonification). 

%Specifically, we will investigate three core hypotheses: i) the ratio of titania to hematite controls the ratio of ammonium/nitrate during fixation and nitrification, ii) optimum nitrogen photofixation occurs at moderate temperatures and humidities (35 C and 50\%) and iii) surface impurities are critical for enabling photocatalytic nitrogen fixation. The knowledge gained from testing these hypotheses will significantly impact the understanding of the global abiotic nitrogen cycle, which has the potential to alter our understanding of soil nutrition in a range of environments. This will be the first work to explore this enigmatic phenomenon with modern techniques, and will provide key insight into the broader importance of the emerging field of photogeochemistry in the natural nutrient cycle. 


%Furthermore,this work will advance knowledge in the fields of surface science and catalysis by establishing a molecular-level understanding of photocatalytic nitrogen based transformation on titania. This will be the first time this system has been studied with applied bias, and the first photocatalytic nitrogen fixation study using in situ spectroscopy (FTIR, XPS). These unique experiments will be coupled with density functional theory (DFT) calculations to enable new atomic-scale insights into the surface intermediates.  Gaining a molecular-level understanding of the photocatalytic nitrogen fixation process on titania will be a foundation step to the rational optimization and design of photocatalysts that can enable more sustainable fertilizer production processes.

\textbf{Broader Impacts:} Managing the nitrogen cycle has been deemed a critical component of the food-energy-water (FEW) nexus, as control over the composition of reduced and oxidized nitrogen species in soils is needed for food production. Furthermore, with growing populations in a range of centralized and decentralized regions, understanding nutrient transformations processes will only grow in importance. The critical first step in managing the nitrogen cycle is to define the magnitude and mechanisms of each N-based transformation. This is historically difficult, as the  nitrogen cycle is known to be the most complex biogeochemical nutrient cycle.  The proposed broader impact of this work will specifically aim to determine how solar-driven abiotic N-transformations contribute to nutrient cycles. 

The impact of the work will be enhanced by outreach and educational efforts. The findings will be highlighted through a publicly accessible website and YouTube videos that demonstrate key findings and suggest simple experiments that can be performed with household items. Furthermore, the work will be the basis of a capstone project through Georgia Tech's Food-Energy-Water initiatives enabled through the Serve-Learn-Sustain program. This capstone project will explore scale-up challenges and engage undergraduate seniors from a range of academic disciplines including engineering and social sciences. The project will seek to establish a holistic view of the role of photocatalytic nitrogen fixation on the current environment and future chemical processes.



% * <marta.hatzell@me.gatech.edu> 2016-10-08T19:22:22.067Z:
%
% we may not want to bring up nitrate and water pollution issues. I get why, but since we are not directly targeting this issue it might be more confusing to a reviewer. the INFEWS call clearly states that alternative fertilizer production is an area of interests, so we should not have to worry about the connection to much.I removed the statement "Over fertilization can also lead to nitrate pollution of the groundwater tables.  

% overall the connections to broader impacts in text are good here"
%
% ^.
%The impact of the work will be enhanced by outreach and educational efforts. The findings will be highlighted through a publicly accessible website and YouTube videos that demonstrate key findings and suggest simple experiments that can be performed with household items. Furthermore, the work will be the basis of a capstone project through Georgia Tech's Food-Energy-Water initiatives enabled through the Serve-Learn-Sustain program. This capstone project will explore scale-up challenges and engage undergraduate seniors from a range of academic disciplines including engineering and social sciences. The project will seek to establish a holistic view of the role of photocatalytic nitrogen fixation on the current environment and future chemical processes.

\newpage
