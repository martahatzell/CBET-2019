\renewcommand{\LeftFooter}{Facilities \& Equipment}
\renewcommand{\PageLimit}{0}
\begin{center}
{\bf Facilities \& Equipment} \\*[3mm]
\end{center}

\setcounter{section}{0}


{\setlength{\parindent}{0cm}The laboratories of Dr. Hatzell’s group consist of 800 ft$^2$ of wet chemical laboratory space with 2 fume hoods and one instrumentation laboratories (ca. 400 ft$^2$). Cubicles for graduate students are located adjacent to the main wet chemical laboratory. Dr. Hatzell's groups experimental lab focuses on photo- and electrochemistry and has the following equipment.}
\begin{itemize}
\item Single channel portable Gamry potentiostat with EIS
\item Multiple channel (8) Biologic VMP3 potentiostat with EIS
\item Pine Rotating Disk Electrode 
\item Pine photoelectrochemical three electrode quartz reactor
\item SRI Gas chromatograph with gas and hydrocarbon analyzers (FID and TCD detectors)
\item Sorvall Centrifuge
\item Four port schlenk line for nanoparticle synthesis
\item Autoclaves for hydrothermal synthesis
\item Vacuum tube furnace with He, Ar, forming gas and nitrogen connections. 
\item 1000 W Xenon UV Lamp (Newport) 
\item Portable Avantas UV/Vis spectroscopy
\end{itemize}
Dr. Medford’s facilities consist of cubicles for graduate students and priority access to the dedicated computational cores on the ChBE subsection of the PACE supercomputing cluster which will be expanded in January 2017 to a total size of ca. 65 24-core 128 GB Intel nodes corresponding to a score of $\sim$50,500 on the SPECfp\_rate2006 floating point performance benchmark. The PACE cluster contains another 8500 shared computational cores along with a data center for secure data storage.

Additional experimental resources are available at the NMR center and the Center for Nanomaterial Characterization (TEM, SEM, DLS). Both centers are located on campus. The School of Chemical \& Biomolecular Engineering has a mechanical workshop. Workshops for glassware and electronics are available in the School of Chemistry \& Biochemistry at Georgia Tech. Machine shops are available through the Mechanical Engineering Department, and free to all faculty and students. Details on shared resources:
\begin{itemize}
\item NMR: Georgia Tech has a well equipped NMR user facility with six spectrometers with field strength between 7 and 11.7 T (1H resonance frequency: 300-500 MHz). Two of the spectrometers are equipped for solid state measurements using magic angle spinning (MAS). One of these spectrometers can also be used for nuclear magnetic imaging. A multitude of different nuclei can be probed and probes with two and three channel allow for conducting sophisticated multi-pulse experiments. Experiments can be performed in a temperature range from -150 C to +250 C. 

\item Scanning Electron Microscopy: Georgia Tech has a Hitachi S800 field emission gun (FEG) scanning electron microscope (SEM) with secondary electron imaging at 3 nm resolution and an EDS spectrometer for elemental analysis down to Fluorine. We also have access to a LEO 1530 thermally-assisted field emission (TFE) scanning electron microscope (SEM), a LEO 1550 thermally-assisted field emission (TFE) scanning electron microscope (SEM). 

\item Transmission electron microscopy: Georgia Tech has a Hitachi HF-2000 field emission gun (FEG) transmission electron microscope (TEM) 200kV thin window EDS spectrometer for quantitative microanalysis parallel-detection electron energy-loss spectrometer (PEELS), and a JEOL 4000EX high resolution electron microscope (HREM) 400kV.

\item X-ray Photoelectron spectroscopy: Thermo Scientific K-Alpha XPS with sputter gun.

\item Raman Spectroscopy: Alpha-Witek confocal Raman microscope with a laser wavelength of 514 nm.

\end{itemize}

